\documentclass[]{article}
\usepackage{lmodern}
\usepackage{amssymb,amsmath}
\usepackage{ifxetex,ifluatex}
\usepackage{fixltx2e} % provides \textsubscript
\ifnum 0\ifxetex 1\fi\ifluatex 1\fi=0 % if pdftex
  \usepackage[T1]{fontenc}
  \usepackage[utf8]{inputenc}
\else % if luatex or xelatex
  \ifxetex
    \usepackage{mathspec}
  \else
    \usepackage{fontspec}
  \fi
  \defaultfontfeatures{Ligatures=TeX,Scale=MatchLowercase}
\fi
% use upquote if available, for straight quotes in verbatim environments
\IfFileExists{upquote.sty}{\usepackage{upquote}}{}
% use microtype if available
\IfFileExists{microtype.sty}{%
\usepackage{microtype}
\UseMicrotypeSet[protrusion]{basicmath} % disable protrusion for tt fonts
}{}
\usepackage[margin=1in]{geometry}
\usepackage{hyperref}
\hypersetup{unicode=true,
            pdftitle={Preliminary results for a statistical catch-at-age model for sablefish (Anoplopoma fimbria) in the Northern Southeast Inside management area},
            pdfborder={0 0 0},
            breaklinks=true}
\urlstyle{same}  % don't use monospace font for urls
\usepackage{longtable,booktabs}
\usepackage{graphicx,grffile}
\makeatletter
\def\maxwidth{\ifdim\Gin@nat@width>\linewidth\linewidth\else\Gin@nat@width\fi}
\def\maxheight{\ifdim\Gin@nat@height>\textheight\textheight\else\Gin@nat@height\fi}
\makeatother
% Scale images if necessary, so that they will not overflow the page
% margins by default, and it is still possible to overwrite the defaults
% using explicit options in \includegraphics[width, height, ...]{}
\setkeys{Gin}{width=\maxwidth,height=\maxheight,keepaspectratio}
\IfFileExists{parskip.sty}{%
\usepackage{parskip}
}{% else
\setlength{\parindent}{0pt}
\setlength{\parskip}{6pt plus 2pt minus 1pt}
}
\setlength{\emergencystretch}{3em}  % prevent overfull lines
\providecommand{\tightlist}{%
  \setlength{\itemsep}{0pt}\setlength{\parskip}{0pt}}
\setcounter{secnumdepth}{0}
% Redefines (sub)paragraphs to behave more like sections
\ifx\paragraph\undefined\else
\let\oldparagraph\paragraph
\renewcommand{\paragraph}[1]{\oldparagraph{#1}\mbox{}}
\fi
\ifx\subparagraph\undefined\else
\let\oldsubparagraph\subparagraph
\renewcommand{\subparagraph}[1]{\oldsubparagraph{#1}\mbox{}}
\fi

%%% Use protect on footnotes to avoid problems with footnotes in titles
\let\rmarkdownfootnote\footnote%
\def\footnote{\protect\rmarkdownfootnote}

%%% Change title format to be more compact
\usepackage{titling}

% Create subtitle command for use in maketitle
\newcommand{\subtitle}[1]{
  \posttitle{
    \begin{center}\large#1\end{center}
    }
}

\setlength{\droptitle}{-2em}

  \title{Preliminary results for a statistical catch-at-age model for sablefish
(\emph{Anoplopoma fimbria}) in the Northern Southeast Inside management
area}
    \pretitle{\vspace{\droptitle}\centering\huge}
  \posttitle{\par}
    \author{Jane Sullivan\(^1\), Ben Williams\(^1\), Andrew Olson\(^2\)\\
\(^1\)Alaska Department of Fish and Game, Commercial Fisheries Division,
Juneau, Alaska\\
\(^2\)Alaska Department of Fish and Game, Commercial Fisheries Division,
Douglas, Alaska}
    \preauthor{\centering\large\emph}
  \postauthor{\par}
    \date{}
    \predate{}\postdate{}
  
\usepackage{booktabs}
\usepackage{longtable}
\usepackage{array}
\usepackage{multirow}
\usepackage[table]{xcolor}
\usepackage{wrapfig}
\usepackage{float}
\usepackage{colortbl}
\usepackage{pdflscape}
\usepackage{tabu}
\usepackage{threeparttable}
\usepackage{threeparttablex}
\usepackage[normalem]{ulem}
\usepackage{makecell}

\usepackage{float} \floatplacement{figure}{H}

\begin{document}
\maketitle

{
\setcounter{tocdepth}{2}
\tableofcontents
}
\section{Introduction}\label{introduction}

Sablefish have been commercially fished in Southeast Alaska inside
waters since at least the early 1900s (Carlile et al. 2002). There is a
long history of sablefish management in the Northern Southeast Inside
(NSEI) management area, with seasonal closures beginning in 1945 and
tagging experiments occurring as early as 1951 (Figure 1; Carlile et al.
2002). Attempts to develop a statistical catch-at-age model for NSEI
sablefish date back to 1995 (Carlile et al. 2002, Dressel 2009, Mueter
2010, Williams and Kirk 2017).

Currently the Alaska Department of Fish and Game (ADF\&G) conducts an
annual mark-recapture pot survey in May that serves as the basis for
stock assessment and managment (Stahl and Holum 2010). Tags are
recaptured in the ADF\&G longline survey in July and the longline
fishery in August (Beder and Stahl 2016). A time-stratified Chapmanized
Petersen model is used to estimate abundance in the Bayesian open source
software \texttt{JAGS\ 4.3.0} (Chapman 1951, Sullivan and Williams 2018,
Depaoli 2016). The abundance estimate is then partitioned into age
classes and biomass estimates using age composition and weight-at-age
data collected during the longline survey and fishery. A
yield-per-recruit model is used to estimate \(F_{50\%}\) using the
\texttt{optim()} function in the statistical software \texttt{R} (R Core
Team 2018). ADF\&G has defined Acceptable Biological Catch (ABC) as
\(F_{ABC}\)=\(F_{50\%}\) for the NSEI sablefish stock (Dressel 2009).

Several factors motivated the development of a new statistical
catch-at-age model. The current ADF\&G framework relies heavily on the
mark-recapture experiment, which may be vulnerable to future budget
cuts. Further the mark-recapture estimate provides a single snapshot in
time and therefore results in high inter-annual variability in abundance
and biomass estimates. Consequently, we are unable to fully integrate
the available data sources, explore historical trends, or adequately
assess stock status or harvest strategies. ADF\&G collects a significant
amount of data in the NSEI through multiple surveys, logbooks, and port
sampling (Figure \ref{fig:datsrc}). Moving to a new modeling framework
will allow us to better utilize these data and will make management more
resilient to potential budget cuts. In addition, the current assessment
relies on Federal estimates of selectivity and does not estimate
recruitment for the stock. If there are differences in availability,
gear selectivity, or stock dynamics in NSEI, we are unable to detect
them. Finally, strong recruitment from the 2014 and possibly 2013 and
2015 year classes were reported in the Federal assessment, prompting
questions about how to treat the uncertainty in recruitment for State
management (Hanselman et al. 2017, Sullivan and Williams 2018). A
statistical catch-at-age model coded in Template Model Builder
(\texttt{TMB}) will allow more flexibility in exploring recruitment
using random effects (Kasper et al. 2016).

\section{Modeling approach}\label{modeling-approach}

The statistical catch-at-age model presented here was coded in
\texttt{TMB}, an \texttt{R} library that leverages \texttt{C/C++}
functionality to calculate first and second order derivatives and was
inspired by a similar \texttt{C/C++} templating software \texttt{ADMB}
(Kasper et al. 2016, Fournier et al. 2012). \texttt{TMB} is preferred to
\texttt{ADMB} for it's easy implementation of random effects (Kasper et
al. 2016). The \texttt{TMB} code replicates or makes refinements to
methods used in a previous attempt at modeling the NSEI sablefish stock
(Mueter 2010); this model was based on \texttt{ADMB} code from an older
Federal assessment of sablefish that has also been adapted for Alaska
rockfish stocks (Kimura 1990, Sigler 1999). The model can be run as
either a single-sex or sex-structured model; however, data inputs are
only shown for the sex-structured option. Variable definitions for all
equations used in the statistical catch-at-age model can be found in
Table \ref{tab:vardefns}.

\section{Data inputs}\label{data-inputs}

The data used as inputs to the \texttt{TMB} model, including point
estimates, variance, and sample sizes for composition data, can be found
in \texttt{seak\_sablefish/data/TMB\_inputs}. A summary of the available
data by year can be found in Figure \ref{fig:datsrc}.

\subsection{Weight-at-age}\label{weight-at-age}

Data from the 2002-2018 longline fishery and 1997-2018 ADF\&G longline
surveys were used to obtain weight-at-age. We fit sex-specific
three-parameter weight-based Ludwig von Bertalanffy growth models to
weight-at-age data:

\begin{equation}
\ln(w_a)=\ln W_{\infty}+\beta \cdot \ln (1-\exp (a-t_0)) ,
\label{eq:weightvonb}
\end{equation}

where \(w_a\) is weight at a given age (kg), \(W_{\infty}\) is the mean
asymptotic weight (kg), \(\beta\) is the power in the allometric
equation and relates to the rate at which \(W_{\infty}\) is reached, and
\(t_0\) is the theoretical age at weight zero (years).

The Federal assessment uses survey weight-at-age exclusively to fit to
catch and effort indices (Hanselman et al. 2018). However, because
discarding is allowed in the State fishery, there are large differences
in survey and fishery weight-at-age, especially at younger ages (Figure
\ref{fig:bioinputs}A). Consequently fishery weight-at-age was used to
fit to landed catch biomass. Survey weight-at-age was used to estimate
derived time series of exploitable biomass, spawning biomass, and other
quantities of interest in the model (Figure \ref{fig:bioinputs}A).

\subsection{Maturity-at-age}\label{maturity-at-age}

Data from the 2007-2018 ADF\&G longline surveys were used to fit a
maturity curve for females and used to estimate spawning stock biomass
within the model (Figure \ref{fig:bioinputs}B). Alternative length, age,
and year-specific models were evaluated using Akaike Information
Criterion (AIC) (Akaike 1974). The length-based maturity curve fit to
all years was the best-fitting model. We used a logistic regression
approach in \texttt{R}, such that the probability \(p\) of being mature
at a given length on the logit scale is a linear function of length
(\(l\)):

\begin{equation}
\ln(\frac{p_l}{1-p_l})=\beta_0+\beta_1 \cdot l.
\label{eq:maturity}
\end{equation}

Predicted maturity at length was transformed to age using the fitted
values from a length-based von Bertalanffy growth curve (Figure
\ref{fig:bioinputs}B). The length at 50\% maturity is 61 cm and the age
at 50\% maturity is 6.4 years. Predicted proportions mature-at-age were
used as inputs to the assessment model and in the calculation of
spawning stock biomass.

\subsection{Sex ratios}\label{sex-ratios}

Using sex ratio data from the 1997-2018 ADF\&G longline surveys, we fit
a generalized additive model (GAM) using the \texttt{gam()} function in
the \texttt{mgcv} R package (Wood 2011). The probability of being
female-at-age \(r_a\) is modeled as a smooth function of age \(a\)

\begin{equation}
\ln(\frac{r_a}{1-r_a})=s(a).
\label{eq:sexratio}
\end{equation}

Fits to the data suggest that female sablefish make up the majority of
catch-at-age in the survey until roughly age-18 and then decline to
\textless{}40\% by age-30 (Figure \ref{fig:bioinputs}C). Predicted
values of proportion female-at-age were used to estimate spawning stock
biomass in the single sex model. These data are not used in the
sex-structured model.

\subsection{Catch}\label{catch}

Catch data from 1980-2018 included harvest in the directed sablefish
longline fishery, ADF\&G longline survey removals, and sablefish
retained in other fisheries like the IFQ halibut longline fishery
(Figure \ref{fig:abdind}A). Catch was assumed to be lognormally
distributed, with a fixed log standard deviation of 0.05.

Changes in the management structure during this period included a move
to Limited Entry in 1985 and the Equal Quota Share (EQS) Program in
1994. Additional sources of mortality that are not currently included in
this model include sport, subsistence and personal use harvest,
estimated bycatch mortality in the halibut fishery, and estimated
deadloss, which includes mortality from sand fleas, sharks, and whales.

\subsection{Fishery CPUE}\label{fishery-cpue}

Fishery catch-per-unit-effort (CPUE) in kg per hook was used as an index
of abundance in the model from 1980-2018 (Figure \ref{fig:abdind}B).
This index was assumed to be lognormally distributed, with a fixed log
standard deviation of 0.1 for the historical data (1980-1996) and 0.08
for the contemporary data (1997 to present). Separate catchabilities and
selectivity curves were assumed for pre-EQS and EQS time periods (Table
\ref{tab:fedsel}).

\subsection{Survey CPUE}\label{survey-cpue}

Longline survey CPUE in numbers per hook was used as an index of
abundance in the model from 1997-2018 (Figure \ref{fig:abdind}C). This
index was assumed to be lognormally distributed, with a fixed log
standard deviation of 0.1. An earlier ADF\&G longline survey in
1988-1996 used a shorter soak time of 1-hr instead of the current 3-11
hr (Carlile et al. 2002). These data were omitted because the 1-hr soak
time was likely too short to provide an accurate measure of relative
abundance (Sigler 1993).

\subsection{Mark-recapture abundance}\label{mark-recapture-abundance}

The mark-recapture abundance index was included for 2003-2010, 2012,
2013, 2015, 2017, and 2018 (Figure \ref{fig:abdind}D). These estimates
of abundance come from a time-stratified modified Petersen
mark-recapture model that was implemented in \texttt{JAGS\ 4.3.0}
(Sullivan and Williams 2018, Depaoli 2016). Further information about
how these indices were derived can be found in Sullivan and Williams
(2018). This index was assumed to be lognormally distributed, and the
log standard deviation was approximated as the coefficient of variation
from the posterior distribution of the mark-recapture abundance
estimates.

\subsection{Age compositions}\label{age-compositions}

Fishery age compositions from the 2002-2018 longline fishery and survey
age compositions from the 1997-2018 ADF\&G longline surveys were
included in the model (Figure \ref{fig:agecomps}). Sample sizes were
deemed insufficient to fit age compositions by sex, so age data have
been aggregated for both the survey and fishery. Age compositions were
assumed to follow the multinomial or Dirichlet-multinomial distributions
(only results for the multinomial are shown in this report). Until more
sophisticated tuning methods or estimates of effective sample size can
be developed for NSEI, effective sample sizes were calculated as the
square root of the total sample size in a given year.

An ageing error matrix for NSEI is currently being developed in
conjunction with the ADF\&G Age Determination Unit. Until this has been
fully developed and reviewed, the Federal sablefish ageing error matrix
has been made available to the State (D. Hanselman, Fisheries Research
Biologist, NOAA, Juneau, personal communication April 2019; Hanselman et
al. 2018; Heifetz et al. 1999; Figure \ref{fig:ageerror}). The ageing
error matrix (\(\Omega_{a',a}\)) is the proportion observed at age \(a\)
given the true age \(a'\). Ageing error matrices are critical for
correcting observed age compositions and estimating recruitment
(Fournier and Archibald 1982).

\subsection{Length compositions}\label{length-compositions}

Length data from the 2002-2018 longline fishery and 1997-2018 ADF\&G
longline surveys were summarized using the Federal conventions for
length compositions (Hanselman et al. 2018). The Federal assessment uses
2-cm length bins ranging from 41-99 cm. Fish less than 41 cm (\(l_0\))
were omitted from the analysis, and fish greater than 99 cm were
aggregated into the 99 cm length bin (\(l_{+}\)). Sample sizes were
adequate to separate length compositions by sex.

Length distributions in fishery (Figure \ref{fig:fshlen}) show
dramatically different patterns than the survey (Figure
\ref{fig:srvlen}), with the fishery length distribution truncated at
approximately 60 cm. Unlike the Federal sablefish fishery, releasing of
fish is allowed in the State fishery. Because smaller sablefish are
worth significantly less per pound than larger individuals, the pattern
of consistently large fish in the fishery length compositions is
attributed to this discarding behavior. The State fishery has no
observer coverage, so fitting to fishery age and length compositions and
estimating fishery selectivity is a technical challenge. Because of the
bias introduced by allowing fish to be released in the fishery, there is
a question of whether fishery age and length compositions should be
included in model.

Finally, the selective harvest of larger-bodied individuals results in
large differences between survey and fishery size-at-age. Until separate
age-length keys can be developed using NSEI data, the Federal age-length
keys (\(\Lambda_{a,l,k}\)) were used to fit both survey and fishery
length compositions (D. Hanselman, Fisheries Research Biologist, NOAA,
Juneau, personal communication April 2019; Hanselman et al. 2018; Echave
et al. 2012; Figure \ref{fig:agelenkey}). Ultimately, separate
age-length keys should be developed for each data source.

\subsection{Retention probability}\label{retention-probability}

To model the discarding behavior in the NSEI fishery, processor grade
and price per pound data were used to inform retention probabilities at
size (Figure \ref{fig:retention}). Based on conversations with ADF\&G
port sampling staff and fishermen, the lower bound of the Grade 2/3 (1.4
kg) was assigned a 10\% retention probability, the lower bound of the
Grade 3/4 (2.2 kg) was assigned a 50\% retention probability, and
everything greater than 6.4 kg was assigned a 100\% retention
probability (A. Olson, Groundfish Project Leader, ADF\&G July 2018).
Remaining retention probabilities were interpolated between these fixed
values. Weight-based retention probabilities were translated to sex and
age \(R_{a,k}\) using sex- and weight-based von Bertalanffy growth
curves (Figure \ref{fig:retention}).

\section{Model parameters}\label{model-parameters}

\subsection{Natural mortality}\label{natural-mortality}

Natural mortality \(M\) was assumed constant over time and age and is
fixed at 0.10, which is consistent with past State and Federal
assessments (Johnson and Quinn 1988, Hanselman et al. 2018).

\subsection{Discard mortality}\label{discard-mortality}

Stachura et al. (2012) estimated discard mortality \(D\) of sablefish to
be 11.7\% using release-recapture data from a longline survey in
Southeast Alaska. It is likely that discard mortality in a fishery is
higher due to careful fish handling on survey vessels during tagging
experiments. We therefore used \(D\)=16\%, the discard mortality rate
from the Pacific halibut fishery (Gilroy and Stewart 2013). The halibut
fishery is assumed a good proxy for sablefish, because the fisheries
utilize similar gear and frequently the same vessels and crew
participate in both fisheries. Moreover, both species are considered
hardy, do not experience barotrauma, and are known to survive well in
laboratory experiments.

\subsection{Selectivity}\label{selectivity}

The longline fishery and survey are assumed to follow a logistic
selectivity pattern. Currently two parameterizations of logistic
selectivity are available in the \texttt{TMB} model.

The first parameterization uses \(s_{50}\) and \(s_{95}\), which
represent the ages at which 50\% and 95\% of individuals are selected by
the gear. Selectivity-at-age (\(s_a\)) for this parameterization is
defined as

\begin{equation}
s_a=\frac{1}{1+\mbox{exp}\frac{-\mbox{ln}(19)(a-s_{50})}{s_{95}-s_{50}}}.
\label{eq:sel1}
\end{equation}

The second parameterization uses \(s_{50}\) and \(\delta\), which
represent the ages at which 50\% of individuals are selected by the gear
and the shape or slope of the logistic curve, respectively.
Selectivity-at-age (\(s_a\)) for this parameterization is defined as

\begin{equation}
s_a=\frac{1}{1+\mbox{exp}(-k(a-s_{50}))}.
\label{eq:sel2}
\end{equation}

Selectivity is fit separately for the longline fishery (\(fsh\)) and
survey (\(srv\)). There is flexibility to define discrete time blocks
for both fishery and survey selectivity.

Currently fishery and survey selectivity are fixed in the model using
Federal selectivity values for the derby (pre-EQS), modern fishery
(EQS), longline survey (Hanselman et al. 2018, Figure
\ref{fig:fixedsel}, Table \ref{tab:fedsel}).

Estimating selectivity will be challenging when accounting for fishery
discards because no age or length data are available on the discarded
population. One potential solution is to estimate a single selectivity
for the longline survey and then apply that selectivity curve to the
fishery. Further information is needed to better characterize how
discarding behavior has changed over time and if discarding was common
pre-EQS.

\subsection{Catchability}\label{catchability}

Currently four parameters for catchability are estimated: two for
fishery catchability (pre-EQS and EQS) \(\text{ln}(q_{fsh})\), one for
the ADF\&G longline survey \(\text{ln}(q_{srv})\), and one for the
mark-recapture abundance index \(\text{ln}(q_{MR})\).

\subsection{Recruitment and initial
numbers-at-age}\label{recruitment-and-initial-numbers-at-age}

The numbers-at-age matrix \(N\) is parameterized with mean
log-recruitment \(\mu_R\), 39 (\(T\)) log-recruitment deviations
\(\tau\), mean log initial numbers-at-age \(\mu_N\), and 28 (\(A-2\))
deviations from mean log initial numbers-at-age \(\psi\).

Following the Federal assessment, if recruitment is estimated using
penalized likelihood, the parameter that describes the variability of
\(\tau\) and \(\psi\), \(\text{ln}(\sigma_R)\), is fixed at 0.1823
(Sigler et al. 2002, Hanselman et al. 2018). However, if \(\tau\) and
\(\psi\) are estimated as random effects, \(\text{ln}(\sigma_R)\) is an
estimated parameter. Results are only shown for the penalized likelihood
approach.

\subsection{Fishing mortality}\label{fishing-mortality}

There is one parameter estimated for mean log-fishing mortality,
\(\mu_F\), and 39 (\(T\)) log-fishing mortality deviations \(\phi\).

\section{Population dynamics}\label{population-dynamics}

The population dynamics of this model are governed by the following
state dynamics equations, where the number of sablefish \(N\) in year
\(t=1\), age \(a\), and sex \(k\) are defined as

\begin{equation}
N_{1,a,k} = \left\{ \begin{array}{ll}
0.5\cdot\mbox{exp}(\mu_R-M(a-a_0)+\psi_a) &a_{0}<a<a_{+}\\
0.5\cdot\mbox{exp}(\mu_R-M(a_{+}-1))/(1-\mbox{exp}(-M)) &a=a_{+}
\end{array}\right.
\label{eq:Nmat1}
\end{equation}

Recruitment to age-2 in all years and the remaining projected \(N\)
matrix is defined as

\begin{equation}
N_{t,a,k} = \left\{ \begin{array}{ll}
0.5\cdot\mbox{exp}(\mu_R+\tau_t) &a=a_0\\
0.5\cdot N_{t-1,a-1,k}\mbox{exp}(Z_{t-1,a-1,k}) &a_{0}<a<a_{+}\\
0.5\cdot N_{t-1,a-1,k}\mbox{exp}(Z_{t-1,a-1,k})+N_{t-1,a,k}\mbox{exp}(Z_{t-1,a,k}) &a=a_{+}
\end{array}\right.
\label{eq:Nmat2}
\end{equation}

where the total instantaneous mortality, \(Z_{t,a,k}\), is the sum of
natural mortality \(M\) and fishing mortality \(F_{t,a,k}\).

Total annual fishing mortality \(F_t\) is defined as

\begin{equation}
F_t=\mbox{exp}(\mu_F+\phi_t).
\label{eq:fmort1}
\end{equation}

Fishing mortality is modeled as a function of fishery selectivity
\(s_{t,a,k}\), retention probability \(R_{a,k}\) (the age-specific
probability of being landed given being caught, Figure
\ref{fig:retention}), and discard mortality \(D\):

\begin{equation}
F_{t,a,k}=s_{t,a,k}^{fsh}(R_{a,k}+D(1-R_{a,k}))F_t.
\label{eq:fmort2}
\end{equation}

\section{Predicted values}\label{predicted-values}

Predicted fishery CPUE (kg per hook) in year \(t\) \(\hat{I_t}^{fsh}\)
was defined as a function of fishery catchability \(q_{fsh}\) and
biomass available to the fishery:

\begin{equation}
\hat{I_t}^{fsh}=q_{fsh}\sum_{k=1}^{2}\sum_{a=a_0}^{a+}w_{a,k}^{srv} \cdot s_{t,a,k}^{fsh} \cdot N_{t,a,k} \cdot S^{fsh},
\label{eq:predfshcpue}
\end{equation}

where \(w_{a,k}^{srv}\) is mean weight-at-age by sex in the longline
survey and \(S^{srv}\) is the fraction of individuals in year \(t\)
surviving to the beginning of the fishery in August. Survival equations
include natural and fishing mortality because the model assumes
continuous fishing mortality.

Predicted longline survey CPUE (numbers per hook) in year \(t\)
(\(\hat{I_t}^{srv}\)) was defined as a function survey catchability
\(q^{srv}\), abundance available to the survey, and survival to the
beginning of the survey in July (\(S^{srv}\)):

\begin{equation}
\hat{I_t}^{srv}=q_{srv}\sum_{k=1}^{2}\sum_{a=a_0}^{a+}s_{t,a,k}^{srv} \cdot N_{t,a,k} \cdot S^{srv}.
\label{eq:predsrvcpue}
\end{equation}

Predicted mark-recapture abundance in year \(t\) (\(\hat{I_t}^{MR}\))
was defined as a function of mark-recapture catchability \(q^{MR}\),
abundance available to the fishery, and survival to the beginning of the
NSEI fishery in August (\(S^{fsh}\)):

\begin{equation}
\hat{I_t}^{MR}=q_{MR}\sum_{k=1}^{2}\sum_{a=a_0}^{a+}s_{t,a,k}^{fsh} \cdot N_{t,a,k} \cdot S^{fsh}.
\label{eq:predmr}
\end{equation}

Spawning biomass \(SSB\) was calculated as

\begin{equation}
SSB=\sum_{a=a_0}^{a+} w_{a,f}^{srv} \cdot N_{t,a,f} \cdot S^{spawn} \cdot p_a,
\label{eq:ssb}
\end{equation}

where \(w_{a,f}^{srv}\) is mean weight-at-age of females in the longline
survey, \(S^{spawn}\) is the fraction of individuals surviving to spawn
in February, and \(p_a\) is the proportion of females mature in the
survey at age. In the single sex model, proportion of females at age in
the survey \(r_a\) is used to get the female portion of the \(N\)
matrix.

Predicted survey age compositions (sexes combined) were computed as

\begin{equation}
\hat{P}_{t,a}^{srv}=\Omega_{a',a}\frac{\sum_{k=1}^{2}N_{t,a,k} \cdot s_{a,k}^{srv}}{\sum_{k=1}^{2}\sum_{a=a_0}^{a+} N_{t,a,k} \cdot s_{a,k}^{srv}},
\label{eq:predsrvage}
\end{equation}

where \(\Omega_{a',a}\) is the ageing error matrix. Predicted fishery
age compositions (sexes combined) were computed as

\begin{equation}
\hat{P}_{t,a}^{fsh}=\Omega_{a',a}\frac{\sum_{k=1}^{2}C_{t,a,k}}{\sum_{k=1}^{2}\sum_{a=a_0}^{a+} C_{t,a,k}},
\label{eq:predfshage}
\end{equation}

where \(C_{t,a,k}\) is the landed catch in numbers-at-age by sex derived
from a modified Baranov catch equation:

\begin{equation}
C_{t,a,k}=N_{t,a,k}\frac{R_{a,k}F_{t,a,k}}{Z_{t,a,k}}(1-\mbox{exp}(-Z_{t,a,k})),
\label{eq:landed}
\end{equation}

where \(R_{a,k}\) is the assumed probability of retention by age and sex
(Figure \ref{fig:retention}).

Predicted landed catch in biomass \(\hat{Y}\) was calculated as the
product of fishery weight-at-age \(w_{a,k}^{fsh}\) and landed catch in
numbers-at-age:

\begin{equation}
\hat{Y}_t=\sum_{k=1}^{2}\sum_{a=a_0}^{a+} w_{a,k}^{fsh} \cdot C_{t,a,k}.
\label{eq:yield}
\end{equation}

The biomass of discarded sablefish estimated to die (\(W_t\)) with an
assumed discard mortality (\(D\)) of 0.16 is

\begin{equation}
W_t= \sum_{k=1}^{2}\sum_{a=a_0}^{a+}w_{a,k}^{srv}N_{t,a,k}\frac{D (1-R_{a,k})F_{t,a,k}}{Z_{t,a,k}}(1-\mbox{exp}(-Z_{t,a,k})).
\label{eq:wastage}
\end{equation}

Predicted survey length compositions were calculated using the
sex-specific age-length keys (\(\Lambda_{a,l,k}\)), such that

\begin{equation}
\hat{P}_{t,l,k}^{srv}=\Lambda_{a,l,k}\frac{N_{t,a,k} \cdot s_{a,k}^{srv}}{\sum_{a=a_0}^{a+} N_{t,a,k} \cdot s_{a,k}^{srv}}.
\label{eq:predsrvlen}
\end{equation}

Fishery length compositions were calculated as

\begin{equation}
\hat{P}_{t,l,k}^{fsh}=\Lambda{a,l,k}\frac{C_{t,a,k}}{\sum_{a=a_0}^{a+} C_{t,a,k}}.
\label{eq:predfshlen}
\end{equation}

\section{Likelihood components}\label{likelihood-components}

The objective function, or the total negative log-likelihood to be
minimized, included the sum of the following likelihood components
\(L\), which received individual weights \(\lambda\):

\begin{enumerate}
\def\labelenumi{\arabic{enumi}.}
\item
  Landed catch biomass (\(Y\)) was modeled using a lognormal likelihood
  where \(\sigma_Y\) was assumed to be 0.05:

  \begin{equation}
  -\mbox{ln}L(Y)=\lambda_Y\frac{1}{2\sigma_Y^2}\sum_{t=1}^{T}\Big(\mbox{ln}(Y_t+c)-\mbox{ln}(\hat{Y}_t+c)\Big)^2 ,
  \label{eq:catchlike}
  \end{equation}

  where \(\lambda_Y\) = 1.0 and \(c\) is a small constant set at 0.0001
  to allow approximately zero catches in log-space.
\item
  Fishery CPUE, survey CPUE, and the mark-recapture abundance index were
  modeled using lognormal likelihoods, where \(\sigma_I\) was assumed to
  be 0.08 for the fishery and survey CPUEs and annual posterior standard
  deviations were used for the mark-recapture abundance index:

  \begin{equation}
  -\mbox{ln}L(I)=\lambda_I\frac{1}{2\sigma_I^2}\sum_{t=1}^{T_I}\Big(\mbox{ln}(I_t+c)-\mbox{ln}(\hat{I}_t+c)\Big)^2 ,
  \label{eq:indexlike}
  \end{equation}

  where \(T_I\) is the number of years of data for each index and
  \(\lambda_I\) is set to 1.0.
\item
  Fishery and survey age compositions were modeled using the multinomial
  likelihood (\(P^{age}\)), where effective sample size \(\omega_t\) was
  calculated as the square root of the total sample size in year \(t\):

  \begin{equation}
  -\mbox{ln}L(P^{age})=\lambda_{P^{age}}\sum_{t=1}^{T_P^{age}} - \omega_t \sum_{a=a_0}^{a+} (P_{t,a}+c)\cdot\mbox{ln}(\hat{P}_{t,a}+c),
  \label{eq:agemult}
  \end{equation}

  where \(T_P^{age}\) is the number of years of data for each age
  composition, \(\lambda_{P^{age}}\) is set to 1.0, and \(c\) prevents
  the composition from being 0 in the likelihood calculation. Standard
  methods of tuning the effective sample size or iterative re-weighting
  have not yet been applied to this assessment model (McAllister and
  Ianelli 1997, Francis 2011). Alternatively, effective sample size can
  be calculated through the estimation of an additional parameter
  \(\theta\) using the Dirichlet-multinomial likelihood (Thorson et al.
  2017):

  \begin{equation}
  -\mbox{ln}L(P^{age})=\sum_{t=1}^{T_P^{age}} -\Gamma(n_t+1)-\sum_{a=a_0}^{a+}\Gamma(n_t P_{t,a}+1)+\Gamma(n_t\theta)-\Gamma(n_t+\theta n_t)+\sum_{a=a_0}^{a+}\Big[\Gamma(n_tP_{t,a}+\theta n_t \hat{P}_{t,a})-\Gamma(\theta n_t \hat{P}_{t,a})\Big],
  \label{eq:agedirich}
  \end{equation}

  where \(n\) is the input sample size. The relationship between \(n\),
  \(\theta\), and \(\omega\) is

  \begin{equation}
  \omega_t = \frac{1+\theta n_t}{1+\theta}.
  \label{eq:effn}
  \end{equation}

  Because the implementation of the alternative Dirichlet-multinomial
  likelihood is currently under development, only results for the
  multinomial likelihood are presented here.
\item
  Fishery and survey length compositions by sex were modeled using the
  multinomial likelihood (\(P^{len}\)), where effective sample size
  \(\omega_t\) was calculated as the square root of the total sample
  size in year \(t\):

  \begin{equation}
  -\mbox{ln}L(P^{len})=\lambda_{P^{len}}\sum_{k=1}^{2}\sum_{t=1}^{T_P^{len}} - \omega_t \sum_{l=l_0}^{l+} (P_{t,l}+c)\cdot\mbox{ln}(\hat{P}_{t,l}+c).
  \label{eq:lenmult}
  \end{equation}

  \(T_P^{len}\) is the number of years of data for each length
  composition and \(\lambda_{Plen}\) is set to 1.0.
\item
  Annual log-fishing mortality deviations (\(\phi_t\)) are included with
  a penalized lognormal likelihood, where

  \begin{equation}
  -\mbox{ln}L(\phi)=\lambda_{\phi}\sum_{t=1}^{T}\phi_t^2,
  \label{eq:fmortlike}
  \end{equation}

  where \(\lambda_{\phi}\)=0.1.
\item
  Recruitment deviations (\(\tau_t\)) can be included using a penalized
  lognormal likelihood

  \begin{equation}
  -\mbox{ln}L(\tau)=\lambda_{\tau}\sum_{i=1}^{T+A-2}(\tau_i-0.5\sigma_R^2)^2,
  \label{eq:reclike}
  \end{equation}

  where \(-0.5\sigma^2\) is a bias correction needed to obtain the
  expected value (mean) instead of the median. The
  \(\lambda_{\phi}\)=2.0 and \(\sigma_R\) is fixed at 1.2 as in the
  Federal assessment (Hanselman et al. 2018). Alternatively, recruitment
  deviations can be estimated as a random effect, where

  \begin{equation}
  -\mbox{ln}L(\tau)=\sum_{i=1}^{T+A-2}\mbox{ln}(\sigma_R)+\frac{(\tau_i-0.5\sigma_R^2)^2}{2\sigma_R}.
  \label{eq:randomrec}
  \end{equation}

  Initial numbers-at-age deviations \(\psi_a\) are implemented in the
  same way as recruitment deviations and are governed by the same
  \(\sigma_R\). Only results for the penalized likelihood approach are
  shown.
\item
  Because the mark-recapture abundance index scales the exploitable
  population, a normal prior is imposed on \(q_{MR}\) of 1.0 with a
  standard deviation of 0.1. Vague priors are assigned to fishery and
  survey \(q\). Future work on this model should include the development
  of priors for fishery and survey \(q\).
\end{enumerate}

\section{Preliminary results and
discussion}\label{preliminary-results-and-discussion}

A summary of parameter estimates and standard errors are reported in
Table \ref{tab:keyparams}. In particular, mean recruitment and
deviations were difficult to estimate. Initially, the weight on this
likelihood component was assumed to be low (\(\lambda_{\tau}=0.1\)), but
this yielded an unrealistically large spike (\textgreater{}40 times the
mean value) in age-2 recruitment in 2016 corresponding to the 2014 year
class. Increasing the \(\lambda_{\phi}\) to 2.0 resulted in more
reasonable parameter estimates and decreased the age-2 recruitment to
\textasciitilde{}8 times mean recruitment. The objective function value
(negative log likelihood) was 1,007.1. The maximum gradient component
was 0.00086, barely passing the minimum convergence criteria of 0.001. A
summary of the contributions of each likelihood component to the total
objective function can be found in Table \ref{tab:likesum}.

Preliminary fits to catch and indices of abundance are shown in Figure
\ref{fig:predabdind}. Results suggest that the model fits catch, pre-EQS
fishery CPUE, and mark-recapture abundance reasonably well in most
years. Modern fishery CPUE (EQS) does not fit well, with long runs of
positive or negative residuals (Figure \ref{fig:residabdind}). The model
performs poorly during the period directly following the implementation
of EQS in 1994 for all indices, including catch (Figure
\ref{fig:residabdind}). Prior to implementing this model for management,
further consideration should be given to which abundance indices that
should be used in the model. For example, because discarding is legal in
NSEI and past logbook data have not required released fish to be
recorded, fishery CPUE may not be a suitable index of abundance.
Starting in 2019, fishermen will be required to provide an estimate of
number of released sablefish by set; however, there will still be no
record of length or weight of these discards. Finally, variability in
catch, survey, and fishery CPUE indices was assumed (Figure
\ref{fig:abdind}). Future enhancements could include estimating this
variability using available data.

Fits to fishery and age compositions are shown in Figures
\ref{fig:fshage} and \ref{fig:srvage}, respectively. Although the model
fits the general shape of the age compositions in most years, there are
poor residual patterns (Figure \ref{fig:residage}). Fits to male and
female fishery length compositions are shown in Figures
\ref{fig:malefshlen} and \ref{fig:femalefshlen}, respectively. Fits to
male and female survey length compositions are shown in Figures
\ref{fig:malesrvlen} and \ref{fig:femalesrvlen}, respectively. Similar
to the age compositions, the model predicts the general shape of the
length compositions in for both the survey and fishery in most years.
Despite this, there are also poor residual patterns in the length
compositions (Figure \ref{fig:residlen}).

There are several caveats to the preliminary fits to composition data.
First, no efforts have been made to externally estimate, tune, or
iteratively re-weight the input effective samples sizes for the
composition data (McAllister and Ianelli 1997, Francis 2011). This
exercise should be completed prior to implementation of this model.
Second, results presented here assume fixed selectivity equal to the
Federal fishery. Because no data on discards exist, it may not be
possible to estimate fishery selectivity while fitting to the
composition data. Stock assessments that account for discarded catch
frequently have observer data and will overcome this challenge through
the estimation of a separate selectivity for discarded catch. An
alternative approach is to estimate survey selectivity and then assume
fishery and survey selectivity are equal. Finally, fishery size-at-age
is larger than survey weight-at-age, especially at younger ages.
Consequently, fits to fishery length compositions may benefit from the
development of separate age-length keys for the NSEI survey and fishery.

Derived indices of age-2 recruitment, female spawning stock biomass, and
exploitable abundance and biomass (i.e.~available to the fishery)
suggest that this stock has been in a period of low productivity since
the mid-1990s (Figure \ref{fig:derivedts}). Recruitment trends are
comparable with Federal values, and estimates of spawning stock biomass,
exploitable biomass, and exploitable abundance are in par with past and
current ADF\&G estimates (Hanselman et al 2018, Sullivan and Williams
2018). A time series of fishing mortality and harvest rate (defined as
the ratio of predicted total catch to exploitable biomass) shows that
peak exploitation occurred in the decade following the transition to
EQS, 1995-2005 (Figure \ref{fig:fishmort}). The model suggests that
harvest rates were more than four times what they are today during this
time period.

Although not currently ready to be considered for management, the
statistical catch-at-age model outlined in this paper is planned to be
presented as a management alternative in 2020. Here we provide a summary
of future developments to this model by priority level.

\subsection{High priority}\label{high-priority}

The following tasks must be completed in order for this model to be
considered for management:

\begin{itemize}
\tightlist
\item
  Complete the development and estimation of management reference
  points.
\item
  Develop rationale for the choice of fishery-dependent data sources to
  include in the model and whether fishery selectivity should be fixed
  or estimated. This relates to the challenge of accounting for
  unobserved discards while estimating fishery selectivity and fitting
  to landed catch compositions.
\item
  Improve weighting methods and tune model to composition data.
\end{itemize}

\subsection{Short-term}\label{short-term}

These tasks should be completed within 1-2 years of implementation. They
are critical components of a well-developed statistical catch-at-age
model.

\begin{itemize}
\tightlist
\item
  Implement Bayesian analysis to evaluate posterior densities of
  estimated and derived quantities of interest.
\item
  Conduct retrospective analysis to determine model performance over
  time.
\item
  Develop framework to conduct projections to evaluate stock status and
  assess risk.
\end{itemize}

\subsection{Long-term:}\label{long-term}

These tasks should be completed within 2-4 years of implementation.
While they are not critical to the implementation of the model, they
will improve model-based inference, understanding of stock dynamics, and
data quality.

\begin{itemize}
\tightlist
\item
  Develop ageing error matrices and age-length keys for NSEI.
\item
  Review indices of abundance. In particular the fishery and survey CPUE
  have little contrast and may not be useful indices of abundance. This
  may include standardizing CPUE through generalized linear or addition
  modeling to account for variables to affect CPUE. It may also include
  developing algorithms to identify trip and set targets and allocating
  total trip landings to set effort.
\item
  Evaluate alternative harvest policies and biological reference points.
\item
  Improve methods for accounting for fishery discards, including
  conducting research to better understand discarding behavior and how
  it has changed over time.
\item
  Develop priors for catchability and other relevant parameters.
\item
  Assess alternative sources of data, especially historical biological
  and catch data (Carlile et al. 2002).
\end{itemize}

\section{Acknowledgements}\label{acknowledgements}

We are grateful to Dana Hanselman, Kari Fenske, and Curry Cunningham
from the NOAA Alaska Fisheries Science Center for their technical
expertise and willingness to share data. In addition, we thank Grant
Adams and Andre Punt from the University of Washington for their
assistance with \texttt{TMB} and modeling. Finally we are grateful to
ADF\&G staff who have collected NSEI sablefish data, maintained
documentation, and worked to improve the conservation and management of
this unique fishery. In particular, we are thankful to Groundfish
Project biologists Mike Vaughn, Kamala Carroll, and Aaron Baldwin for
their in depth knowledge of the fishery and available data.

\section{References}\label{references}

Akaike, H. 1974. A new look at the statistical model identification.
IEEE Transactions on Automatic Control 19:716--723.

Beder, A., J. Stahl. 2016. Northern Southeast Inside Commercial
Sablefish Fishery and Survey Activities in Southeast Alaska, 2015.
Alaska Department of Fish and Game, Fishery Management Report No. 15-27,
Anchorage, Alaska.

Carlile, D. W., Richardson, B., Cartwright, M., and O'Connell, V.M.
2002. Southeast Alaska sablefish stock assessment activities 1988--2001,
Alaska Department of Fish and Game, Division of Commercial Fisheries
Juneau, Alaska.

Chapman, D. G. 1951. Some properties of the hypergeometric distribution
with applications to zoological census. University of California
Publications in Statistics 1:131--160.

Depaoli, S., James P. Clifton, and Patrice R. Cobb. 2016. Just Another
Gibbs Sampler (JAGS) Flexible Software for MCMC Implementation. Journal
of Educational and Behavioral Statistics 41.6: 628-649.

Dressel, S.C. 2009. 2006 Northern Southeast Inside sablefish stock
assessment and 2007 forecast and quota. Alaska Department of Fish and
Game, Fishery Data Series No. 09-50, Anchorage, Alaska.

Echave, K. B., D. H. Hanselman, M. D. Adkison, M. F. Sigler. 2012.
Inter-decadal changes in sablefish, Anoplopoma fimbria, growth in the
northeast Pacific Ocean. Fish. Bull. 210:361-374.

Fournier, D. and C. P. Archibald. 1982. A general theory for analyzing
catch at age data. Can. J. Fish. Aq. Sci. 39: 1195-1207.

Fournier, D. A., H. J. Skaug, J. Ancheta, J. Ianelli, A. Magnusson, M.N.
Maunder, A. Nielsen, and J. Sibert. 2012. AD Model Builder: using
automatic differentiation for statistical inference of highly
parameterized complex nonlinear models. Optim. Methods Softw. 27,
233-249.

Francis, R. I. C. C., 2011. Data weighting in statistical fisheries
stock assessment models. Can. J. Fish. Aquat. Sci. 68, 1124--1138.

Hanselman, D. H., C. J. Rodgveller, K. H. Fenske, S. K. Shotwell, K. B.
Echave, P. W. Malecha, and C. R. Lunsford. 2018. Chapter 3: Assessment
of the sablefish stock in Alaska. In: Stock assessment and fishery
evaluation report for the groundfish resources of the GOA and BS/AI as
projected for 2019. North Pacific Fishery Management Council, 605 W 4th
Ave, Suite 306 Anchorage, AK 99501.

Hanselman, D. H., C. J. Rodgveller, C. R. Lunsford, and K. H Fenske.
2017. Chapter 3: Assessment of the sablefish stock in Alaska. In: Stock
assessment and fishery evaluation report for the groundfish resources of
the GOA and BS/AI as projected for 2018. North Pacific Fishery
Management Council, 605 W 4th Ave, Suite 306 Anchorage, AK 99501.

Heifetz, J., D. Anderl, N.E. Maloney, and T.L. Rutecki. 1999. Age
validation and analysis of ageing error from marked and recaptured
sablefish, Anoplopoma fimbria. Fish. Bull. 97: 256-263.

Johnson, S. L., and T. J. Quinn II. 1988. Catch-Age Analysis with
Auxiliary Information of sablefish in the Gulf of Alaska. Contract
report to National Marine Fisheries Service, Auke Bay, Alaska. 79
pp.~Center for Fisheries and Ocean Sciences, University of Alaska,
Juneau, Alaska.

Kimura, D. K. 1990. Approaches to age-structured separable sequential
population analysis. Can. J. Fish. Aquat. Sci. 47: 2364-2374.

Kristensen, K., A. Nielsen, C. W. Berg, H. Skaug, B. M. Bell. 2016. TMB:
Automatic Differentiation and Laplace Approximation. Journal of
Statistical Software, 70(5), 1-21.\url{doi:10.18637/jss.v070.i05}.

McAllister, M. K., Ianelli, J. N., 1997. Bayesian stock assessment using
catch-age data and the sampling: importance resampling algorithm. Can.
J. Fish. Aquat. Sci. 54,284--300.

Mueter, F. 2010. Evaluation of stock assessment and modeling options to
assess sablefish population levels and status in the Northern Southeast
Inside (NSEI) management area. Alaska Department of Fish and Game,
Special Publication No. 10-01, Anchorage, Alaska.

Sigler, M. F. 1993. Stock assessment and management of sablefish
Anoplopoma fimbria in the Gulf of Alaska. PhD Dissertation. University
of Washington. 188 pp.

Sigler, M. F., 1999. Estimation of sablefish, Anoplopoma fimbria,
abundance off Alaska with an age-structured population model. Fishery
Bulletin, 97: 591-603.

Sigler, M. F., C. R. Lunsford, J. T. Fujioka, and S. A. Lowe. 2002.
Alaska sablefish assessment for 2003. In Stock assessment and fishery
evaluation report for the groundfish fisheries of the Bering Sea and
Aleutian Islands. pp.~449-514. North Pacific Fishery Management Council,
605 W 4th Avenue, Suite 306, Anchorage, AK 99510.

Thorson, J. T., Johnson, K. F., Methot, R. D., \& Taylor, I. G. 2017.
Model-based estimates of effective sample size in stock assessment
models using the Dirichlet-multinomial distribution. Fisheries Research,
192, 84-93.

Wood, S. N. 2011. Fast stable restricted maximum likelihood and marginal
likelihood estimation of semiparametric generalized linear models.
Journal of the Royal Statistical Society (B) 73(1):3-36.

\section{Tables}\label{tables}

\begin{table}

\caption{\label{tab:vardefns}Variable definitions for the statistical catch-at-age model.}
\centering
\begin{tabular}[t]{ll}
\toprule
Variable & Definition\\
\midrule
\addlinespace[0.3em]
\multicolumn{2}{l}{\textbf{$\textit{Indexing and model dimensions}$}}\\
\hspace{1em}$T$ & Number of years in the model\\
\hspace{1em}$t$ & Index for year in model equations\\
\hspace{1em}$A$ & Number of ages in the model\\
\hspace{1em}$a$ & Index for age in model equations\\
\hspace{1em}$a_0$ & Recruitment age (age-2)\\
\hspace{1em}$a_{+}$ & Plus group age (age-31)\\
\hspace{1em}$l$ & Index for length bin in model equations\\
\hspace{1em}$l_0$ & Recruitment length bin (41 cm)\\
\hspace{1em}$l_{+}$ & Plus group length bin (99 cm)\\
\hspace{1em}$fsh$ & NSEI longline fishery\\
\hspace{1em}$srv$ & ADF\&G longline survey\\
\hspace{1em}$MR$ & Mark-recapture abundance\\
\addlinespace[0.3em]
\multicolumn{2}{l}{\textbf{$\textit{Parameters}$}}\\
\hspace{1em}$M$ & Instantaneous natural mortality\\
\hspace{1em}$F$ & Instantaneous fishing mortality\\
\hspace{1em}$Z$ & Total instantaneous mortality\\
\hspace{1em}$S$ & Total annual survival\\
\hspace{1em}$D$ & Discard mortality\\
\hspace{1em}$s_{50}$ & Age at which 50\% of individuals are selected to the gear\\
\hspace{1em}$s_{95}$ & Age at which 95\% of individuals are selected to the gear\\
\hspace{1em}$\delta$ & Slope parameter in the logistic selectivity curve\\
\hspace{1em}$q$ & Catchability\\
\hspace{1em}$\mu_R$ & Mean log recruitment\\
\hspace{1em}$\tau_t$ & Log recruitment deviations\\
\hspace{1em}$\mu_N$ & Mean log initial numbers-at-age\\
\hspace{1em}$\psi_a$ & Log deviations of initial numbers-at-age\\
\hspace{1em}$\sigma_R$ & Variability in recruitment and initial numbers-at-age\\
\hspace{1em}$\mu_F$ & Mean log fishing mortality\\
\hspace{1em}$\phi_t$ & Log fishing mortality deviations\\
\hspace{1em}$\theta$ & Dirichlet-multinomial parameter related to effective sample size\\
\addlinespace[0.3em]
\multicolumn{2}{l}{\textbf{$\textit{Data and predicted variables}$}}\\
\hspace{1em}$w_a$ & Weight-at-age\\
\hspace{1em}$p_a$ & Proportion mature-at-age\\
\hspace{1em}$r_a$ & Proportion female-at-age\\
\hspace{1em}$R$ & Retention probability\\
\hspace{1em}$s_a$ & Selectivity-at-age\\
\hspace{1em}$\Omega_{a',a}$ & Ageing error matrix (proportion observed at age given the true age $a'$)\\
\hspace{1em}$\Lambda_{a,l,k}$ & Age-length key (proportion in length bin given age and sex)\\
\hspace{1em}$N$ & Numbers-at-age\\
\hspace{1em}$C$ & Landed catch in numbers-at-age\\
\hspace{1em}$I$, $\hat{I}$ & Indices of abundance, $\hat{I}$ are predicted values\\
\hspace{1em}$P_a$, $\hat{P}_a$ & Age compositions, $\hat{P}_a$ are predicted values\\
\hspace{1em}$P_l$, $\hat{P}_l$ & Length compositions, $\hat{P}_l$ are predicted values\\
\hspace{1em}$Y$, $\hat{Y}$ & Landed catch biomass, $\hat{Y}$ are predicted values\\
\hspace{1em}$\hat{W}$ & Estimated mortality from discards (biomass)\\
\hspace{1em}$\lambda$ & Weight for likelihood component\\
\hspace{1em}$L$ & Likelihood\\
\hspace{1em}$\omega$ & Effective sample size for age and length compositions\\
\hspace{1em}$n$ & Input sample size for Dirichlet-multinomial likelihood\\
\hspace{1em}$c$ & Small constant (0.00001)\\
\bottomrule
\end{tabular}
\end{table}

\begin{table}

\caption{\label{tab:fedsel}Assumed selectivity parameters for the fishery before the Equal Quota Share program started in 1994 (pre-EQS), the fishery since the implementation of EQS, and the ADF\&G longline survey for females (black points) and males (grey triangles). These parameters estimates were borrowed from the Federal stock assessment, where the Federal derby fishery, IFQ fishery, and NMFS Cooperative Longline Survey were assumed to represent pre-EQS, EQS, and the ADF\&G longline survey (Hanselman et al. 2018).}
\centering
\begin{tabular}[t]{lllll}
\toprule
\multicolumn{1}{c}{ } & \multicolumn{2}{c}{Male} & \multicolumn{2}{c}{Female} \\
\cmidrule(l{2pt}r{2pt}){2-3} \cmidrule(l{2pt}r{2pt}){4-5}
 & $s_{50}$ & $\delta_{50}$ & $s_{50}$ & $\delta_{50}$\\
\midrule
Pre-EQS Fishery & 5.12 & 2.57 & 2.87 & 2.29\\
EQS Fishery & 4.22 & 2.61 & 3.86 & 2.61\\
Longline survey & 3.72 & 2.21 & 3.75 & 2.21\\
\bottomrule
\end{tabular}
\end{table}

\begin{table}

\caption{\label{tab:keyparams}Parameter estimates from the statistical catch-at-age model. Estimates of recruitment, initial numbers-at-age, and fishing mortality deviations were excluded for brevity.}
\centering
\begin{tabular}[t]{lll}
\toprule
Parameter & Estimate & Standard error\\
\midrule
Pre-EQS catchability, $\mbox{ln}(q_{fsh,pre-EQS})$ & -17.618 & 0.044\\
EQS catchability, $\mbox{ln}(q_{fsh,EQS})$ & -16.911 & 0.024\\
Survey catchability, $\mbox{ln}(q_{srv})$ & -16.276 & 0.023\\
Mark-recapture catchability, $\mbox{ln}(q_{MR})$ & -0.038 & 0.010\\
Mean log recruitment, $\mu_R$ & 6.224 & 0.093\\
\addlinespace
Mean log initial numbers-at-age, $\mu_N$ & 6.561 & 0.127\\
Mean log fishing mortality, $\mu_F$ & -2.601 & 0.359\\
\bottomrule
\end{tabular}
\end{table}

\begin{table}

\caption{\label{tab:likesum}Negative likelihood values and percent of each component to the total likelihood.}
\centering
\begin{tabular}[t]{lll}
\toprule
Likelihood component & $NLL$ & \% of $NLL$\\
\midrule
Catch & 13.1 & 1.3\\
Fishery CPUE & 133.6 & 13.3\\
Survey CPUE & 52.0 & 5.2\\
Mark-recapture abundance & 52.0 & 5.2\\
Survey ages & 181.1 & 18.0\\
\addlinespace
Fishery ages & 146.2 & 14.5\\
Survey lengths & 107.7 & 10.7\\
Fishery lengths & 284.7 & 28.3\\
Data likelihood & 970.5 & 96.4\\
Total likelihood & 1007.1 & 100.0\\
\bottomrule
\end{tabular}
\end{table}

\pagebreak

\section{Figures}\label{figures}

\begin{figure}
\centering
\includegraphics{../figures/tmb/sable_data_all.png}
\caption{\label{fig:datsrc}A summary of the available data sources in NSEI
by year.}
\end{figure}

\begin{figure}
\centering
\includegraphics{../figures/tmb/bio_dat.png}
\caption{\label{fig:bioinputs}Biological inputs to the statistical
catch-at-age model, including: (A) weight-at-age (kg) by sex from the
longline fishery (black) and ADF\&G longline survey (grey); (B)
proportion mature for females estimated from the longline survey with
the age at 50\% maturity (\(a_{50}\)=6.4 yr); and (C) proportion female
in the longline survey, where the curve is the fitted line from a
generalized additive model +/- 2 standard error.}
\end{figure}

\begin{figure}
\centering
\includegraphics{../figures/tmb/abd_indices_2018.png}
\caption{\label{fig:abdind}Indices of catch and abundance with the assumed
error distribution, including: (A) harvest (round mt), (B) fishery catch
per unit effort in round kg per hook, (C) survey catch per unit effort
in number of fish per hook, and (D) mark-recapture abundance estimates
in millions. The dashed vertical line in 1994 mark the transition to the
Equal Quota Share program.}
\end{figure}

\begin{figure}
\centering
\includegraphics{../figures/tmb/agecomps_2018.png}
\caption{\label{fig:agecomps}Proportions-at-age for in the NSEI longline
fishery (2002-2018) and ADF\&G longline survey (1997-2018).}
\end{figure}

\begin{figure}
\centering
\includegraphics{../figures/tmb/ageing_error.png}
\caption{\label{fig:ageerror}Ageing error matrix used in the model, showing
the probability of observing an age given the true age (Heifetz et al.
1999).}
\end{figure}

\begin{figure}
\centering
\includegraphics{../figures/tmb/lencomp_fsh_2018.png}
\caption{\label{fig:fshlen}Fishery length distributions by sex from
2002-2018. The dashed vertical line at 61 cm represents the length at
50\% maturity.}
\end{figure}

\begin{figure}
\centering
\includegraphics{../figures/tmb/lencomp_srv_2018.png}
\caption{\label{fig:srvlen}Longline survey length distributions by sex from
1997-2018. The dashed vertical line at 61 cm represents the length at
50\% maturity.}
\end{figure}

\begin{figure}
\centering
\includegraphics{../figures/tmb/age_length_key.png}
\caption{\label{fig:agelenkey}Age-length key used in the model, showing the
probability that a fish of a given age falls within a certain length bin
(Echave et al. 2012).}
\end{figure}

\begin{figure}
\centering
\includegraphics{../figures/retention_prob_2018.png}
\caption{\label{fig:retention}The probability of retaining a fish as a
function of weight (left), sex, and age (right).}
\end{figure}

\begin{figure}
\centering
\includegraphics{../figures/tmb/fixed_selectivity_2018.png}
\caption{\label{fig:fixedsel}Fixed age-based selectivity curves for the
fishery before the Equal Quota Share program started in 1994 (pre-EQS),
the fishery since the implementation of EQS, and the ADF\&G longline
survey for females (black points) and males (grey triangles). These
parameters estimates were borrowed from the Federal stock assessment for
the derby fishery (pre-EQS), IFQ fishery (EQS), and NMFS Cooperative
Longline Survey (Hanselman et al. 2018).}
\end{figure}

\begin{figure}
\centering
\includegraphics{../figures/tmb/pred_abd_indices.png}
\caption{\label{fig:predabdind}Fits to indices of catch and abundance with
the assumed error distribution shown as shaded grey polygons. Input data
are shown as grey points and model fits are shown in black. Indices
include (A) harvest (round mt); (B) fishery catch per unit effort in
round kg per hook with separate selectivity and catchability time
periods before and after the implementation of the Equal Quota Share
program in 1994; (C) survey catch per unit effort in number of fish per
hook; and (D) mark-recapture abundance estimates in millions. Solid and
dashed lines in panel D reflect years for which data were and were not
available, respectively.}
\end{figure}

\begin{figure}
\centering
\includegraphics{../figures/tmb/presid_abd_indices.png}
\caption{\label{fig:residabdind}Standardized residuals of fits to indices of
catch and abundance, including: (A) harvest, (B) fishery catch per unit
effort, (C) survey catch per unit effort, and (D) mark-recapture (MR)
abundance.}
\end{figure}

\begin{figure}
\centering
\includegraphics{../figures/tmb/Fishery_agecomps_barplot.png}
\caption{\label{fig:fshage}Fits to fishery age compositions, 2002-2018.
Observed and predicted proportions-at-age shown as grey bars and black
lines, respectively.}
\end{figure}

\begin{figure}
\centering
\includegraphics{../figures/tmb/Survey_agecomps_barplot.png}
\caption{\label{fig:srvage}Fits to survey age compositions, 1997-2018.
Observed and predicted proportions-at-age shown as grey bars and black
lines, respectively.}
\end{figure}

\begin{figure}
\centering
\includegraphics{../figures/tmb/agecomps_residplot.png}
\caption{\label{fig:residage}Standardized residuals of fits to fishery
(2002-2018) and survey (1997-2018) age compositions. Size of residual
scales to point size. Black points represent negative residuals
(observed \textless{} predicted); white points represent positive
residuals (observed \textgreater{} predicted).}
\end{figure}

\begin{figure}
\centering
\includegraphics{../figures/tmb/Fishery_Male_lencomps_barplot.png}
\caption{\label{fig:malefshlen}Fits to male fishery length compositions,
2002-2018. Observed and predicted proportions-at-age shown as grey bars
and black lines, respectively.}
\end{figure}

\begin{figure}
\centering
\includegraphics{../figures/tmb/Fishery_Female_lencomps_barplot.png}
\caption{\label{fig:femalefshlen}Fits to female fishery length compositions,
2002-2018. Observed and predicted proportions-at-age shown as grey bars
and black lines, respectively.}
\end{figure}

\begin{figure}
\centering
\includegraphics{../figures/tmb/Survey_Male_lencomps_barplot.png}
\caption{\label{fig:malesrvlen}Fits to male survey length compositions,
1997-2018. Observed and predicted proportions-at-age shown as grey bars
and black lines, respectively.}
\end{figure}

\begin{figure}
\centering
\includegraphics{../figures/tmb/Survey_Female_lencomps_barplot.png}
\caption{\label{fig:femalesrvlen}Fits to female survey length compositions,
1997-2018. Observed and predicted proportions-at-age shown as grey bars
and black lines, respectively.}
\end{figure}

\begin{figure}
\centering
\includegraphics{../figures/tmb/lencomps_residplot.png}
\caption{\label{fig:residlen}Standardized residuals of fits to fishery
(2002-2018) and survey (1997-2018) length compositions for males and
females. Size of residual scales to point size. Black points represent
negative residuals (observed \textless{} predicted); white points
represent positive residuals (observed \textgreater{} predicted).}
\end{figure}

\begin{figure}
\centering
\includegraphics{../figures/tmb/derived_ts.png}
\caption{\label{fig:derivedts}Model predictions of (A) age-2 recruitment
(millions), (B) female spawning stack biomass (million lb), (C)
exploitable abundance (millions), and (D) exploitable biomass (million
lb).}
\end{figure}

\begin{figure}
\centering
\includegraphics{../figures/tmb/fishing_mort.png}
\caption{\label{fig:fishmort}Model-estimated fishing mortality rate (top)
and realized harvest rate (bottom), defined as the ratio of total
predicted catch to exploitable biomass. Total predicted catch is the sum
of landed catch and discarded biomass assumed to die post-release.}
\end{figure}


\end{document}
